\documentclass[12pt,a4paper]{article}
\usepackage[utf8]{inputenc}
\usepackage[T1]{fontenc}
\usepackage{fontawesome5}
\usepackage[french]{babel}
\usepackage{geometry}
\usepackage{graphicx}
\usepackage{array}
\usepackage{titlesec}
\usepackage{longtable}
\usepackage{hyperref}
\geometry{margin=2.5cm}
\usepackage{fancyhdr}
\usepackage{subcaption}
\usepackage{enumitem}
\usepackage{colortbl}  % Pour rowcolors dans les tableaux
\usepackage[table]{xcolor}  % Couleurs avancées pour tableaux

% Support pour les images SVG
\usepackage{svg}
\svgsetup{inkscapelatex=false}

% Packages pour les sections écologie
\usepackage{tcolorbox}
\tcbuselibrary{skins}
\usepackage{tikz}
\usetikzlibrary{shapes.geometric, positioning, shadows.blur, calc, arrows.meta}
\usepackage{xcolor}
\usepackage{needspace}

% Couleurs écologiques
\definecolor{ecoVert}{HTML}{27AE60}
\definecolor{ecoVertFonce}{HTML}{1E8449}
\definecolor{ecoVertClair}{HTML}{A9DFBF}
\definecolor{ecoBleu}{HTML}{3498DB}
\definecolor{ecoMarron}{HTML}{795548}
\definecolor{ecoFond}{HTML}{F0FDF4}
\definecolor{traitBleu}{HTML}{2C3E50}
\definecolor{ecoRouge}{HTML}{FF0000}

% Style tcolorbox pour les sous-sections méthodologie (adaptatif)
% Éviter les ruptures de page à l'intérieur des boîtes
\newtcolorbox{methodebox}[2][]{
    enhanced,
    colback=white,
    colframe=#2,
    fonttitle=\bfseries,
    coltitle=white,
    colbacktitle=#2,
    rounded corners,
    boxrule=1.2pt,
    left=6pt, right=6pt, top=6pt, bottom=6pt,
    before={\needspace{10cm}},
    #1
}

\setlength{\parindent}{0pt}
\setlength{\parskip}{0.5em}

\titleformat{\section}{\large\bfseries}{\thesection}{1em}{}
\titleformat{\subsection}{\normalsize\bfseries}{\thesubsection}{1em}{}
% Ajustement de la hauteur d'en-tête pour éviter les avertissements
\setlength{\headheight}{80pt}
\addtolength{\topmargin}{-35pt}

\pagestyle{fancy}
\fancyhf{}
\fancyhead[C]{
  \hspace*{-1in}\hspace*{-\oddsidemargin}
  \includegraphics[width=\paperwidth, height=3cm]{../images/entete.png}
}

% --- NUMÉROTATION DES PAGES ---
\fancyfoot[C]{\thepage}
\renewcommand{\headrulewidth}{0pt}
\renewcommand{\footrulewidth}{0.4pt}

\begin{document}

\thispagestyle{empty}

% PAGE DE GARDE
\begin{titlepage}
    % Titre centré
    \begin{center}
        {\Huge \textbf{MÉMOIRE TECHNIQUE}} \\
        \vspace{0.25cm}
        {\Large \textbf{  }} \\
        \vspace{0.15cm}
        {\large \textbf{LOT : CHARPENTE BOIS }} \\
    \end{center}
    
    \vspace{0.15cm}
    
    % Maître d'ouvrage et adresse chantier
    \begin{center}
    {\small \textbf{Maître d'ouvrage :}  \\
    \textbf{Adresse du chantier :} }
    \end{center}
    
    
    \vspace{0.15cm}
    \begin{center}
    \begin{tcolorbox}[
        enhanced,
        colback=white,
        colframe=ecoBleu!50,
        boxrule=1pt,
        arc=3pt,
        width=0.9\textwidth,
        halign=center
    ]
    
    \includegraphics[width=0.85\textwidth, height=7cm, keepaspectratio]{ ../images/exemple_pagegarde.jpeg }
    
    \end{tcolorbox}
    \end{center}
    
    
    \vspace{0.15cm}
    
    % En-tête avec logo bois et logos de certifications
    \begin{center}
    \includegraphics[height=3.5cm]{../images/logo_boisTechniques.png}
    
    \vspace{0.15cm}
    
    % Une ligne de 8 logos
    \includegraphics[height=1.1cm]{../images/logo_pefc2 (1).png}\hspace{0.5cm}
    \includegraphics[height=1.1cm]{../images/logo_fsc.png}\hspace{0.5cm}
    \includegraphics[height=1.1cm]{../images/logo_rge (1).png}\hspace{0.5cm}
    \includegraphics[height=1.1cm]{../images/logo_qualibat (1).png}\hspace{0.5cm}
    \includegraphics[height=1.1cm]{../images/logo_garantie-decenale2 (1).png}\hspace{0.5cm}
    \includegraphics[height=1.1cm]{../images/logo_ffb_adherent (1).png}\hspace{0.5cm}
    \includegraphics[height=1.1cm]{../images/logo_artisan.png}
    \includegraphics[height=1.1cm]{../images/logo_region_alsace.png}
    \end{center}
    
    \vspace{0.05cm}
    \noindent\rule{\linewidth}{1.5pt}
    
    \vspace{0.05cm}
    
    % Blocs d'informations séparés sans bordures - 3 blocs compacts
    \begin{center}
    \begin{minipage}[c]{5.2cm}
    \centering
    {\fontsize{7.5}{8.5}\selectfont\textbf{ADRESSE :}} \\
    {\fontsize{7}{8}\selectfont ZI - 19 Rue de l'Industrie} \\
    {\fontsize{7}{8}\selectfont 68360 SOULTZ}
    \end{minipage}%
    \hspace{0.08cm}%
    \begin{minipage}[c]{5.2cm}
    \centering
    {\fontsize{7}{8}\selectfont\textbf{TELEPHONE :}} \\
    {\fontsize{7}{8}\selectfont 03 89 53 36 58} \\
    {\fontsize{7}{8}\selectfont\textbf{COURRIEL :}} \\
    {\fontsize{7}{8}\selectfont contact@bois-techniques.fr}
    \end{minipage}%
    \hspace{0.08cm}%
    \begin{minipage}[c]{5.2cm}
    \centering
    {\fontsize{7}{8}\selectfont\textbf{SITE INTERNET :}} \\
    {\fontsize{7}{8}\selectfont www.bois-techniques.fr}
    \end{minipage}
    \end{center}
    
    \vspace{0.05cm}
    \noindent\rule{\linewidth}{0.5pt}
    \vspace{0.05cm}
    
    {\fontsize{6.5}{7.5}\selectfont SAS au capital de 100 000 euros – TVA intracommunautaire FR 50 893 822 841 – Siret n°893 822 841 00027 – APE 4391A – IBAN FR76 3000 4026 7700 0102 3762 986}
    
\end{titlepage}

\tableofcontents
\newpage

% PRÉAMBULE
\section*{PRÉAMBULE}
\noindent
L'entreprise Bois \& Techniques, située à Soultz (68360), est spécialisée dans la
conception, la fabrication et la pose de charpentes bois, ouvrages structurels. Forte d'une expérience significative dans le domaine de la construction
bois, elle intervient sur des projets variés : logements individuels, bâtiments collectifs,
équipements publics ou tertiaires. Restauration de structure bois type colombage et solivage.
\vspace{0.3cm}

L'entreprise dispose de ses propres équipes de charpentiers, encadrées par
un chef de chantier expérimenté. Les ouvrages sont fabriqués dans un atelier équipé
et adaptés aux exigences de la construction bois moderne.
\vspace{0.3cm}

Réalisé par le bureau d’étude interne de la société Bois \& Techniques, le présent mémoire
technique a pour objectif de vous exposer les solutions techniques et l'organisation de
chantier envisagée, afin de mener votre projet dans les meilleures conditions.
En espérant que ce mémoire technique illustrera l'intérêt que nous portons à votre projet.

% SECTIONS DYNAMIQUES ISSUES DU CSV



% Section spéciale pour le contexte du projet (affichage stylisé adaptatif)
\section{ CONTEXTE DU PROJET }














% Plan de masse / Vue aérienne

\begin{tcolorbox}[
    enhanced,
    colback=white,
    colframe=ecoBleu,
    fonttitle=\bfseries\large,
    coltitle=white,
    title={\faIcon{map-marked-alt}\hspace{0.3em}Plan de masse / Vue aérienne},
    colbacktitle=ecoBleu,
    rounded corners,
    boxrule=2pt,
    left=12pt, right=12pt, top=8pt, bottom=8pt
]
\centering

\includegraphics[width=0.90\textwidth, height=10cm, keepaspectratio]{ ../images/vue_aerienne.png }

\end{tcolorbox}
\vspace{0.4cm}

\newpage
\


% Affichage par défaut com pointillés
\begin{tcolorbox}[width=\textwidth,
    enhanced,
    colback=ecoFond,
    colframe=ecoBleu,
    fonttitle=\bfseries\large,
    coltitle=white,
    title={\faIcon{calendar-check}\hspace{0.3em}Visite de site},
    colbacktitle=ecoBleu,
    rounded corners,
    boxrule=2pt,
    left=12pt, right=12pt, top=8pt, bottom=8pt
]
Nous sommes rendus sur les lieux le \dotfill

\vspace{0.3cm}

Adresse : \dotfill

\end{tcolorbox}
\vspace{0.4cm}



\begin{tcolorbox}[width=\textwidth, height=\dimexpr\textheight-6cm\relax,
    enhanced,
    colback=white,
    colframe=ecoBleu,
    fonttitle=\bfseries\large,
    coltitle=white,
    title={\faIcon{file-signature}\hspace{0.3em}Attestation de visite},
    colbacktitle=ecoBleu,
    rounded corners,
    boxrule=2pt,
    left=8pt, right=8pt, top=6pt, bottom=6pt
]
\centering

\includegraphics[width=0.95\textwidth, height=\dimexpr\textheight-8cm\relax, keepaspectratio]{ ../images/attestation_visite.png }

\end{tcolorbox}
\vspace{0.4cm}



\begin{tcolorbox}[
    enhanced,
    colback=white,
    colframe=ecoBleu,
    fonttitle=\bfseries,
    coltitle=white,
    title={\faIcon{clock}\hspace{0.3em}Respect des délais du planning prévisionnel},
    colbacktitle=ecoBleu,
    rounded corners,
    boxrule=1.5pt,
    left=10pt, right=10pt, top=6pt, bottom=6pt
]
Lors de l'étude du planning transmis dans le cadre du dossier de consultation, nous avons porté une attention particulière à la durée prévue par le planning pour la phase de montage.

Au regard des caractéristiques techniques du projet, du volume des éléments à assembler, des conditions d'accès au site, ainsi que de nos retours d'expérience sur des opérations similaires, cette durée apparaît insuffisante pour garantir une exécution conforme aux exigences de qualité, de sécurité et de coordination des intervenants.

Nos prévisions internes, fondées sur une simulation détaillée, estiment le besoin réel à durée insuffisante. Nous recommandons donc d'ajuster le planning initial sur cette base afin d'assurer une exécution réaliste, fluide et sans tension sur les ressources.
\end{tcolorbox}
\vspace{0.4cm}









% Section spéciale pour la situation administrative (affichage stylisé)
\newpage
\section{ SITUATION ADMINISTRATIVE DE L’ENTREPRISE }
% Template stylisé pour la section Situation Administrative de l'Entreprise
% Présentation professionnelle des qualifications, effectifs et chiffres

\vspace{0.3cm}

% Qualifications RGE QUALIBAT
\begin{tcolorbox}[
    enhanced,
    colback=ecoFond,
    colframe=ecoBleu,
    fonttitle=\bfseries\large,
    coltitle=white,
    title={\faIcon{award}\hspace{0.3em}Qualifications entreprise RGE QUALIBAT},
    colbacktitle=ecoBleu,
    rounded corners,
    boxrule=2pt,
    left=12pt, right=12pt, top=10pt, bottom=10pt
]

\begin{itemize}[leftmargin=1.5em, itemsep=4pt]

    \item[\textcolor{ecoBleu}{\faIcon{check-circle}}] Fabrication et pose de charpente traditionnelle

    \item[\textcolor{ecoBleu}{\faIcon{check-circle}}] Fabrication et pose de bâtiments à ossature bois

    \item[\textcolor{ecoBleu}{\faIcon{check-circle}}] Réparation et renforcement d'ouvrage de charpente

    \item[\textcolor{ecoBleu}{\faIcon{check-circle}}] Restauration de charpente du patrimoine ancien

\end{itemize}

\end{tcolorbox}

\vspace{0.5cm}

% Deux colonnes : Effectif et Chiffre d'affaires
\noindent
\begin{minipage}[t]{0.48\textwidth}
\begin{tcolorbox}[
    enhanced,
    colback=white,
    colframe=ecoBleu,
    fonttitle=\bfseries,
    coltitle=white,
    title={\faIcon{users}\hspace{0.3em}Effectif au 01/01/2025},
    colbacktitle=ecoBleu,
    rounded corners,
    boxrule=1.5pt,
    halign=center,
    height=4.5cm
]

\vspace{0.8cm}
{\Huge\bfseries\textcolor{ecoBleu}{ 12 }}

\vspace{0.3cm}
{\large salariés }

\end{tcolorbox}
\end{minipage}
\hfill
\begin{minipage}[t]{0.48\textwidth}
\begin{tcolorbox}[
    enhanced,
    colback=white,
    colframe=ecoMarron,
    fonttitle=\bfseries,
    coltitle=white,
    title={\faIcon{chart-line}\hspace{0.3em}Chiffre d'affaires},
    colbacktitle=ecoMarron,
    rounded corners,
    boxrule=1.5pt,
    left=10pt, right=10pt,
    height=4.5cm
]

\vspace{0.3cm}
\renewcommand{\arraystretch}{1.6}
\begin{tabular}{@{}l r@{}}

    \textbf{ 2022 } & 2 324 861 € \\

    \textbf{ 2023 } & 2 914 663 € \\

    \textbf{ 2024 } & 3 227 324 € \\

\end{tabular}

\vspace{0.3cm}
{\small\textcolor{ecoVert}{ \faIcon{arrow-up} Croissance continue }}

\end{tcolorbox}
\end{minipage}





\newpage
\section{ MOYENS HUMAINS AFFECTES AU PROJET }

    
    
    
    % Inclusion du fichier spécial pour sécurité et santé
    \subsection{ SECURITE ET SANTE SUR LES CHANTIERS }
    % ============================================
% SECURITE ET SANTE SUR LES CHANTIERS
% ============================================

\begin{tcolorbox}[
    enhanced, colback=white, colframe=ecoBleu, fonttitle=\bfseries\small, coltitle=white,
    title={\faIcon{cog}\hspace{0.2em}Organisation de production},    colbacktitle=ecoBleu, rounded corners, boxrule=1pt, left=4pt, right=4pt, top=3pt, bottom=3pt
]
\footnotesize

Notre atelier est organisé de façon lean management en privilegiant l'efficience afin de faciliter la circulation des pièces de bois au cours des différentes phases de travail. Les zones de stockage, de traçage, d'usinage et de stockage avant expédition sont prévues.

\end{tcolorbox}

\vspace{0.4cm}

\begin{tcolorbox}[
    enhanced, colback=white, colframe=ecoBleu, fonttitle=\bfseries\small, coltitle=white,
    title={\faIcon{user-plus}\hspace{0.2em}Accueil des nouveaux salariés},
    colbacktitle=ecoBleu, rounded corners, boxrule=1pt, left=4pt, right=4pt, top=3pt, bottom=3pt
]
\footnotesize
Chaque nouvel arrivant (stagiaire, intérimaire, salarié) se voit fournir, dès son arrivée, un livret d'accueil rappelant les bases de sécurité et les bons comportements à avoir (OPP BTP), une « fiche d'accueil » est rédigée avec lui, afin de faciliter son suivi médical en cas d'accident et de le sensibiliser aux dangers du chantier et les types d'EPI mis à sa disposition.
\end{tcolorbox}

\vspace{0.4cm}

\begin{tcolorbox}[
    enhanced, colback=white, colframe=ecoBleu, fonttitle=\bfseries\small, coltitle=white,
    title={\faIcon{certificate}\hspace{0.2em}Habilitations et compétences réglementaires},    colbacktitle=ecoBleu, rounded corners, boxrule=1pt, left=4pt, right=4pt, top=3pt, bottom=3pt
]
\footnotesize

L'ensemble du personnel intervenant sur les chantiers dispose des habilitations et certifications réglementaires nécessaires à l'exécution des travaux en toute sécurité. Les équipes sont régulièrement formées et maintenues à niveau afin de garantir la conformité aux exigences en vigueur et la maîtrise des risques liés aux interventions.

À ce titre, les opérateurs sont notamment titulaires des habilitations suivantes :
\begin{itemize}[leftmargin=1.5em, itemsep=2pt]
    \item CACES Levage
    \item CACES Conduite de pont roulant
    \item CACES Nacelle
    \item SST (Sauveteur Secouriste du Travail)
\end{itemize}

\end{tcolorbox}

\vspace{0.4cm}

\begin{tcolorbox}[
    enhanced, colback=white, colframe=ecoBleu, fonttitle=\bfseries\small, coltitle=white,
    title={\faIcon{coffee}\hspace{0.2em}Confort de travail},    colbacktitle=ecoBleu, rounded corners, boxrule=1pt, left=4pt, right=4pt, top=3pt, bottom=3pt
]
\footnotesize

Que ce soit en atelier ou sur chantier, nous mettons à la disposition de notre personnel \textbf{une salle de repos} avec : point d'eau, machine à café, réfrigérateur, micro-onde, table et chaises, vestiaire, douche, sanitaires\ldots

\end{tcolorbox}

\vspace{0.4cm}

\begin{tcolorbox}[
    enhanced, colback=ecoFond, colframe=ecoRouge, fonttitle=\bfseries, coltitle=white,
    title={\faIcon{shield-alt}\hspace{0.3em}Sécurité et santé sur les chantiers},
    colbacktitle=ecoRouge, rounded corners, boxrule=1.5pt, left=5pt, right=5pt, top=4pt, bottom=4pt
]
\small
Afin d'offrir des garanties de sécurité au public et nos salariés, notre atelier, \textbf{construit en 2010}, respecte l'ensemble des normes de construction et de prévention/lutte/accès incendie et explosion. Nos équipes ont à leur disposition les moyens de travailler en sécurité tout en préservant leur santé. Ils participent régulièrement à des formations sur les bonnes pratiques sécuritaires et le bon usage de nos machines (tailles et levages).
\end{tcolorbox}

\vspace{0.4cm}

\begin{tcolorbox}[
    enhanced, colback=white, colframe=ecoRouge, fonttitle=\bfseries\small, coltitle=white,
    title={\faIcon{tasks}\hspace{0.2em}Concrètement},    colbacktitle=ecoRouge, rounded corners, boxrule=1pt, left=4pt, right=4pt, top=3pt, bottom=3pt
]
\footnotesize

\begin{itemize}[leftmargin=*, itemsep=0pt, parsep=0pt, topsep=0pt]
    \item NORMES et CERTIFICATIONS MACHINES (fixes et portatives).
    \item NORMES ASPIRATION.
    \item SECURITE INCENDIE. (Prévention, lutte, accès).
    \item NORMES ELECTRIQUE.
    \item ACCESSIBILITE AUX PERSONNES EN SITUATION DE HANDICAP.
    \item HYGIENE, SANTE et CONFORT.
    
\end{itemize}

\end{tcolorbox}

\vspace{0.4cm}

\begin{tcolorbox}[
    enhanced, colback=white, colframe=ecoRouge, fonttitle=\bfseries\small, coltitle=white,
    title={\faIcon{check-double}\hspace{0.2em}Vérification périodique},    colbacktitle=ecoRouge, rounded corners, boxrule=1pt, left=4pt, right=4pt, top=3pt, bottom=3pt
]
\footnotesize

Échelles, harnais, stop-chute, équipement de levage, ligne de vie, installation électrique, détection incendie, RIA, compresseur, chariot élévateur, extincteur, machine atelier\ldots

\end{tcolorbox}

\vspace{0.4cm}

\begin{tcolorbox}[
    enhanced, colback=white, colframe=ecoRouge, fonttitle=\bfseries\small, coltitle=white,
    title={\faIcon{exclamation-triangle}\hspace{0.2em}La sécurité s'affiche partout},
    colbacktitle=ecoRouge, rounded corners, boxrule=1pt, left=4pt, right=4pt, top=3pt, bottom=3pt
]
\footnotesize
De nombreuses affiches issues des organismes de prévention reconnus, tels que l'INRS et l'OPPBTP, sont disposées aux endroits stratégiques de nos bureaux et de notre atelier.

Puisque la pédagogie est l'art de la répétition nous avons privilégié des modèles avec des graphismes et slogan plus ancien pour rappeler à nos charpentiers celle qui se trouvaient déjà autour d'eux dans leur lieu de formation (collège, lycée, CFA,\ldots).
\end{tcolorbox}

    

    
    
    
    % Inclusion du PDF de l'organigramme
    \subsection{ ORGANIGRAMME FONCTIONNEL DE L'EQUIPE AFFECTEE AUX PROJETS }
    
    \begin{center}
    \includegraphics[width=0.85\textwidth, height=0.7\textheight, keepaspectratio]{ ../images/organigramme.png }
    \end{center}
    
    
    

    





% Section spéciale pour les moyens matériels (affichage stylisé)
\newpage
\section{ MOYENS MATERIEL AFFECTES AU PROJET }
\vspace{-0.3cm}
% ============================================
% MOYENS MATERIEL AFFECTES AU PROJET
% ============================================

\begin{tcolorbox}[
    enhanced, colback=ecoFond, colframe=ecoBleu, fonttitle=\bfseries, coltitle=white,
    title={\faIcon{tools}\hspace{0.3em}Un parc machines récent, certifié et entretenu},
    colbacktitle=ecoBleu, rounded corners, boxrule=1.5pt, left=6pt, right=6pt, top=4pt, bottom=4pt
]
\small
L'entreprise Bois \& Techniques dispose d'un parc matériel complet et régulièrement entretenu, permettant d'assurer la qualité et la sécurité de nos interventions.
\end{tcolorbox}

\vspace{0.4cm}

% Conception
\begin{tcolorbox}[
    enhanced, colback=white, colframe=ecoBleu, fonttitle=\bfseries\footnotesize, coltitle=white,
    title={\faIcon{drafting-compass}\hspace{0.2em}Conception et Précision},
    colbacktitle=ecoBleu, rounded corners, boxrule=1pt, left=4pt, right=4pt, top=3pt, bottom=3pt
]
\footnotesize
\begin{itemize}[leftmargin=*, itemsep=1pt, topsep=2pt]
    \item Utilisation d'un scan laser 3D pour le relevé des cotes, des altimétries, de l'implantation et du traçage.
    \item Panel de logiciel (CAO, DAO, calcul de structure, …)
    \item Purge complémentaire
    \item Relevé sur site de l’état sanitaire des pièces conservées.
    \item Plans 3D et 2D (Plans P.A.C).
    
\end{itemize}
\end{tcolorbox}

\vspace{0.4cm}

% Securite (Rouge)
\begin{tcolorbox}[
    enhanced, colback=white, colframe=ecoRouge, fonttitle=\bfseries\footnotesize, coltitle=white,
    title={\faIcon{hard-hat}\hspace{0.2em}Sécurité},
    colbacktitle=ecoRouge, rounded corners, boxrule=1pt, left=4pt, right=4pt, top=3pt, bottom=3pt
]
\footnotesize
\begin{itemize}[leftmargin=*, itemsep=1pt, topsep=2pt]
    \item EPI (Harnais, casque, lignes de vie, stop-chutes, chaussures, lunette, gant, masque)
    \item Filets de protection antichute
    \item Echafaudage roulant intérieur (Hussor, Distel)
    \item Echafaudages extérieurs ponctuels (Hussor)
    \item Nacelle / Echelle d'accès (Distel)
    
\end{itemize}
\end{tcolorbox}

\vspace{0.4cm}

% Atelier (Marron)
\begin{tcolorbox}[
    enhanced, colback=white, colframe=ecoMarron, fonttitle=\bfseries\footnotesize, coltitle=white,
    title={\faIcon{warehouse}\hspace{0.2em}Atelier de taille},
    colbacktitle=ecoMarron, rounded corners, boxrule=1pt, left=4pt, right=4pt, top=3pt, bottom=3pt
]
\footnotesize
\begin{itemize}[leftmargin=*, itemsep=1pt, topsep=2pt]
    \item Scie radiale / Raboteuse Utis
    \item Compresseurs et outillage pneumatique
    \item Scie à panneaux Streibig / Scie à ruban
    \item Radiale Lyonflex / Scie à format / Toupie
    \item Machines de charpente Mafell
    \item Pont roulant 3,2T / Chariot latéral 4T
    
\end{itemize}
\end{tcolorbox}

\vspace{0.4cm}

% Levage
\begin{tcolorbox}[
    enhanced, colback=white, colframe=ecoBleu, fonttitle=\bfseries\footnotesize, coltitle=white,
    title={\faIcon{arrows-alt}\hspace{0.2em}Levage},
    colbacktitle=ecoBleu, rounded corners, boxrule=1pt, left=4pt, right=4pt, top=3pt, bottom=3pt
]
\footnotesize
\begin{itemize}[leftmargin=*, itemsep=1pt, topsep=2pt]
    \item Palan / Grue mobile d'intérieur 600Kg
    \item Vérin 10 tonnes
    \item Tour d'étayage Hussor
    \item Treuil électrique 250Kg
    \item Lève matériaux (chèvre, lève-plaque) 300Kg
    
\end{itemize}
\end{tcolorbox}

\vspace{0.4cm}

% Transport
\begin{tcolorbox}[
    enhanced, colback=white, colframe=ecoBleu, fonttitle=\bfseries\footnotesize, coltitle=white,
    title={\faIcon{truck}\hspace{0.2em}Transport},
    colbacktitle=ecoBleu, rounded corners, boxrule=1pt, left=4pt, right=4pt, top=3pt, bottom=3pt
]
\footnotesize
\begin{itemize}[leftmargin=*, itemsep=1pt, topsep=2pt]
    \item Camionnettes 3,5 Tonnes équipées
    \item Camion bras pour approvisionnement
    \item Remorque
    
\end{itemize}
\end{tcolorbox}

\vspace{0.4cm}

% Machine portative (Marron)
\begin{tcolorbox}[
    enhanced, colback=white, colframe=ecoMarron, fonttitle=\bfseries\footnotesize, coltitle=white,
    title={\faIcon{cogs}\hspace{0.2em}Machine portative},
    colbacktitle=ecoMarron, rounded corners, boxrule=1pt, left=4pt, right=4pt, top=3pt, bottom=3pt
]
\footnotesize
\begin{itemize}[leftmargin=*, itemsep=1pt, topsep=2pt]
    \item Machines portatives de nos équipes (vingtaine par véhicule)
    \item Matériel d'injection pour résine époxy
    
\end{itemize}
\end{tcolorbox}

\vspace{0.4cm}

% Protection
\begin{tcolorbox}[
    enhanced, colback=white, colframe=ecoBleu, fonttitle=\bfseries\footnotesize, coltitle=white,
    title={\faIcon{broom}\hspace{0.2em}Proctection / Nettoyage},
    colbacktitle=ecoBleu, rounded corners, boxrule=1pt, left=4pt, right=4pt, top=3pt, bottom=3pt
]
\footnotesize
\begin{itemize}[leftmargin=*, itemsep=1pt, topsep=2pt]
    \item Bâchage permanent
    \item Aspirateur de chantier
    \item Equipement de nettoyage dans chaque véhicule
    
\end{itemize}
\end{tcolorbox}

\vspace{0.4cm}

% Gestion Dechet (Vert)
\begin{tcolorbox}[
    enhanced, colback=white, colframe=ecoVertFonce, fonttitle=\bfseries\footnotesize, coltitle=white,
    title={\faIcon{recycle}\hspace{0.2em}Gestion déchet},
    colbacktitle=ecoVertFonce, rounded corners, boxrule=1pt, left=4pt, right=4pt, top=3pt, bottom=3pt
]
\footnotesize
\begin{itemize}[leftmargin=*, itemsep=1pt, topsep=2pt]
    \item Big-bag
    \item 6 Bennes de tri chez Bois \& Techniques
    \item Remorque
    
\end{itemize}
\end{tcolorbox}





\newpage
\section{ MÉTHODOLOGIE / CHRONOLOGIE }

    
    
    
    % Style spécial pour déroulement des travaux
    \begin{methodebox}[title={\faIcon{clipboard-list}\hspace{0.3em}Deroulement des travaux  "Restauration"}]{ecoBleu}
    
    Méthode de préparation du chantier : \\
L'entreprise Bois \& Techniques réalisera un relevé précis des charpentes en place avec détails des assemblages, sections, côte altimétriques, zones dégradées ainsi qu'un dossier photos qui permettront de respecter les configurations d'origine de la restauration. \\
D'autre part des notes de calculs seront réalisées afin d'adapter les sections aux différentes charges de couverture, plafond et d'exploitations en accord avec le Maïtre d'oeuvre. \\
Les bois de remplacement proviennent du massif forestier vosgien avec certificat de traitement et PEFC et sont séchés à l'air avant utilisation. \\
Les assemblages seront réalisés conformément aux existants ou selon détails ci après : (INSERER IMAGE) \\
 \\
Nous avons également un stock de bois ancien en sapin et en chêne afin d’obtenir une parfaite harmonie avec les zones visibles. \\
Les bois de charpente à remplacer, à greffer, à résiner, à étayer seront représentés sur nos plans avec les détails d’assemblages et transmis au Maître d’ouvrage pour accord avant travaux. \\
Nous réalisons les renforcements par résine époxy armée depuis 25 ans avec le même fournisseur (MAPEI) en respectant leur cahier de charges, notre personnel est habitué et formé à cette technique de réparation. \\
\textbf{Nous possédons une garantie décennale spécifique pour ces travaux.} \\
Un reportage photos sera réalisé en cours des travaux et envoyé au Maître d'œuvre et d'ouvrage à sa demande. \\
Un relevé contradictoire sera réalisé en fin de phase et envoyé à votre économiste avec notre situation de travaux.
    
    \end{methodebox}
    

    
    
    
    % Style spécial pour conception
    \begin{methodebox}[title={\faIcon{drafting-compass}\hspace{0.3em}Conception}]{ecoBleu}
    
    Méthode de préparation du chantier :  \\
Une collaboration inter-lots est indispensable afin de soulever et de solutionner tous les problèmes techniques et temporels du chantier.  \\
L’entreprise délivre ses plans d’exécution des ouvrages, Plans d’Atelier et de chantier (PAC), calepins et épures, notes de calcul, descente de charge, au bureau de contrôle.   \\
L’entreprise fournit également tous les renseignements indispensables à l’établissement d’un ordonnancement général définissant les phases des études d’exécution y compris PAC, d’exécution des travaux préparatoires des études de synthèse, des mises au point de fabrication, des commandes de matériel, de présentation des échantillons, des formalités administratives, le PLANNING de pose avec sous-phase, la fabrication atelier, …
    
    \end{methodebox}
    

    
    
    
    % Style spécial pour fabrication/taille
    \begin{methodebox}[title={\faIcon{hammer}\hspace{0.3em}Fabrication / Taille en atelier}]{ecoMarron}
    
    
    LES OPERATIONS REALISEES POUR CE PROJET :
    
    
    \end{methodebox}
    

    
    
    
    % Style spécial pour transport et levage
    % Contenu Transport et Levage
    \begin{methodebox}[title={\faIcon{truck-loading}\hspace{0.3em}Transport et Levage}]{ecoBleu}
    
    
    Le chantier est posé par nos équipes formées, chaque chef d'équipe est titulaire de plusieurs « CACES » pour pouvoir manipuler nacelles ciseaux, bras articulé, chariot élévateur… \\
 \\
Les operations realisees pour ce projet : \\
 \\
\vspace{0.3cm} \\
\noindent \\
Ouvrages livrés sur chantier :  \\
Les ouvrages livrés sur le chantier en attente de pose seront stockés à l'abri des intempéries, des souillures et des chocs. Les conditions de stockage seront telles qu'elles ne subissent aucune déformation ou détérioration
    
    
    \end{methodebox}
    \vspace{0.3cm}
    
    % Image de la grue, centrée
    \begin{center}
    \begin{tcolorbox}[
        enhanced,
        colback=white,
        colframe=ecoBleu,
        fonttitle=\bfseries\large,
        coltitle=white,
        title={\faIcon{truck-monster}\hspace{0.3em}Moyens de levage},
        colbacktitle=ecoBleu,
        rounded corners,
        boxrule=2pt,
        width=\textwidth,
        halign=center
    ]
    
    \includegraphics[width=0.8\textwidth, height=8cm, keepaspectratio]{ ../images/grue.png }
    
    \end{tcolorbox}
    \end{center}
    
    

    
    
    
    % Style spécial pour protection
    \begin{methodebox}[title={\faIcon{shield-alt}\hspace{0.3em}Protection de l'existant ou ses ouvrages pour ce projet}]{ecoBleu}
    
    \begin{itemize}
    \item Travaux effectués avec des cloueurs pneumatiques, permettant de limiter les chocs et les vibrations.
    \item Utilisation de compresseurs à haute pression, avec une vitesse de rotation lente, un fonctionnement silencieux et une absence de vibrations.
    \item Préférence pour le vissage au clouage.
\end{itemize}
    
    \end{methodebox}
    

    
    
    
    % Style spécial pour hygiène et sécurité
    \begin{methodebox}[title={\faIcon{first-aid}\hspace{0.3em}Organisation en matière d'hygiène et de sécurité}]{ecoRouge}
    
    
    - Un échafaudage périphérique (hors lot) avec une recette pour l'approvisionnement des matériaux. \\
- Une plateforme provisoire sous les zones de travaux.  \\
- Echafaudage mobile ponctuel intérieur et/ou ligne de vie avec harnais et top-chute.
    
    
    \end{methodebox}
    

    
    
    
    \subsection{ PROTECTION / NETTOYAGE }
    
    Sols - Murs - Bâchage permanent du chantier (selon attribut du lot) - Filet
    

    
    

    
    
    
    % Style spécial pour Respect des contraintes liées aux monuments historiques
    \begin{methodebox}[title={\faIcon{building}\hspace{0.3em}Respect des contraintes liées aux monuments historiques}]{ecoMarron}
    
    
    Les moyens mis en oeuvre pour obtenir :  \\
-  Le label BBC ; \\
-  La certification ; \\
-  Tests d’étanchéité ;   \\
-  Tests acoustiques ;  \\
-  Calculs TH ; \\
-  Respect monument historique
    
    
    \end{methodebox}
    

    
    
    
    % Inclusion du fichier spécial pour la démarche HQE
    % ============================================
% DEMARCHE HQE - Texte exact de la base de données
% ============================================

\subsection{DÉMARCHE HQE}

\begin{tcolorbox}[
    enhanced,
    colback=ecoFond,
    colframe=ecoVert,
    fonttitle=\bfseries\large,
    coltitle=white,
    title={\faLeaf\hspace{0.5em}Hygiène qualité environnement sécurité},
    colbacktitle=ecoVertFonce,
    rounded corners,
    boxrule=2pt,
    left=8pt, right=8pt, top=8pt, bottom=8pt
]

Nous sommes labélisés RGE : Notre démarche HQE (Haute qualité environnementale) tend à offrir des ouvrages sains et confortables dont les impacts sur l'environnement, évalués sur l'ensemble du cycle de vie, sont les plus maîtrisés possibles. C'est une démarche d'optimisation multicritères qui s'appuie sur une donnée fondamentale : tout bâtiment doit avant tout répondre à un usage et assurer un cadre de vie adéquat à ses utilisateurs. Comme l'atteste nos efforts réalisés dans le choix de nos fournisseurs, nos matériaux, la gestion de nos déchets et leurs retraitements, la société Bois \& Techniques agis concrètement pour améliorer la qualité environnementale de nos constructions neuves et des bâti existant.

\end{tcolorbox}

\vspace{0.5cm}

% ---- ECO-CONSTRUCTION ----
\begin{tcolorbox}[
    enhanced,
    colback=white,
    colframe=ecoVert,
    fonttitle=\bfseries,
    coltitle=white,
    title={\faTree\hspace{0.3em}ECO-Construction},
    colbacktitle=ecoVert,
    rounded corners,
    boxrule=1.5pt,
    left=8pt, right=8pt, top=8pt, bottom=8pt
]


\textbf{\textcolor{ecoVertFonce}Cible n°02 « choix intégré des procédés et produits de construction »}

\vspace{0.4cm}


\textbf{- Durabilité, entretien et adaptabilité des bâtiments:}

Nous adaptons nos choix constructifs à la durée de vie souhaitée de l'ouvrage.
Nous prenons en compte l'adaptabilité de l'ouvrage dans le temps et son démontage/séparation des produits et des systèmes constructifs.
Nous prenons en compte la facilité d'accès pour l'entretien du bâti.
Nous privilégions des produits de construction faciles à entretenir.

\vspace{0.3cm}

\textbf{- Choix des procédés de construction:}

Nous choisissons des produits et des systèmes constructifs dont les caractéristiques sont vérifiées.
Nous optimisons la manutention sur le chantier et nous prenons en considération la santé des intervenants sur les chantiers (manutentionnaires, poseurs) afin de prévenir et limiter les troubles musculosquelettiques.

\vspace{0.3cm}

\textbf{- Choix des produits de construction:}

Les choix des matériaux sont compatibles avec une gestion durable de l'environnement (Label Vert, PEFC, FSC, fournisseurs locaux, matériaux certifié Excell Zone Verte (contrôle substance indésirables ambiance intérieur, panneau classe E1) …etc.).

\vspace{0.3cm}



\vspace{0.5cm}


\textbf{\textcolor{ecoVertFonce}Cible n°03 « chantier à faibles nuisances »}

\vspace{0.4cm}


\textbf{- Gestion différenciée des déchets de chantier:}

VOIR : NOTRE DEMARCHE ENVIRONNEMENTALE SUR LES CHANTIERS

\vspace{0.3cm}

\textbf{- Réduction des pollutions de la parcelle et du voisinage:}

Par l'usage de système de vissage en remplacement du clouage.
Nos systèmes constructifs ne nécessitent pas d'eau sur le chantier, évitant ainsi une contamination des sols.
Concentration des moyens de levage pour limiter les nuisances sur le trafic routier environnant.

\vspace{0.3cm}





\end{tcolorbox}

\vspace{0.5cm}

% ---- ECO-GESTION ----
\begin{tcolorbox}[
    enhanced,
    colback=white,
    colframe=ecoBleu,
    fonttitle=\bfseries,
    coltitle=white,
    title={\faCog\hspace{0.3em}ECO-Gestion},
    colbacktitle=ecoBleu,
    rounded corners,
    boxrule=1.5pt,
    left=8pt, right=8pt, top=8pt, bottom=8pt
]


\textbf{\textcolor{ecoBleu}Cible n°04 « gestion de l'énergie »}




\textbf{- Réduction de la demande et des besoins énergétiques:}

Offrir une limitation des perditions énergétiques par les qualités isolantes des systèmes constructifs et matériaux employés.

\vspace{0.3cm}



\vspace{0.5cm}


\textbf{\textcolor{ecoBleu}Cible n°06 « gestion des déchets d'activités »}




\textbf{- Gestion différenciée des déchets d'activités:}

VOIR : NOTRE DEMARCHE ENVIRONNEMENTALE ATELIER ET BUREAUX

\vspace{0.3cm}

\textbf{- Réduction de la production de déchets:}

Faible quantité de déchets produits pendant la phase de construction et de mise en œuvre grâces à nos systèmes constructifs.

\vspace{0.3cm}





\end{tcolorbox}

\vspace{0.5cm}

% ---- CONFORT ----
\begin{tcolorbox}[
    enhanced,
    colback=white,
    colframe=ecoBleu,
    fonttitle=\bfseries,
    coltitle=white,
    title={\faHome\hspace{0.3em}Confort},
    colbacktitle=ecoBleu,
    rounded corners,
    boxrule=1.5pt,
    left=8pt, right=8pt, top=8pt, bottom=8pt
]


\textbf{\textcolor{ecoBleu}Cible n°08 « confort hygrothermique »}

\vspace{0.2cm}


\textbf{- Gestion des zonages hygrothermique et permanence des conditions de confort hygrothermique:}

\vspace{0.2cm}

Attention toutes particulière sur la mise en œuvre du pare vapeur et le respect des consignes de la maîtrise d'œuvre.
Formation RGE de notre bureau d'étude.
Le confort hygrométrique sera assuré par l'utilisation de matériaux de construction respirants.
Les ponts thermiques entre la structure en béton et les systèmes constructifs seront évités par des détails de construction.
Adaptation des matériaux aux pièces à fort dégagement d'humidité (salle d'eaux, sous-sol…).

\vspace{0.4cm}



\vspace{0.4cm}


\textbf{\textcolor{ecoBleu}Cible n°09 « confort acoustique »}

\vspace{0.2cm}


\textbf{- Correction / isolation acoustique et affaiblissement des bruits:}

\vspace{0.2cm}

Nous réduisons efficacement l'apparition de bruits parasite par l'ajout de matière absorbante.
L'amélioration acoustique est prise en compte dans nos conceptions de structure.

\vspace{0.4cm}





\end{tcolorbox}

\vspace{0.5cm}

% ---- SANTE ----
\begin{tcolorbox}[
    enhanced,
    colback=white,
    colframe=ecoRouge,
    fonttitle=\bfseries,
    coltitle=white,
    title={\faHeart\hspace{0.3em}Santé},
    colbacktitle=ecoRouge,
    rounded corners,
    boxrule=1.5pt,
    left=8pt, right=8pt, top=8pt, bottom=8pt
]


\textbf{Cible n°14 « qualité de l'air »}

\vspace{0.2cm}


\textbf{- Gestion des risques de pollution par les produits de construction:}

\vspace{0.2cm}

Afin de limiter le dégagement des particules fines reconnues nuisible pour la santé nous utilisons des bois certifiés, des colles à base aqueuse ainsi que les traitements de classe 2 (label vert -- SARPECO 850) (tous nos panneaux OSB sont certifiés de classe E1).




\end{tcolorbox}
    

    
    
    
    % Inclusion du fichier spécial pour la démarche environnementale atelier
    % ============================================
% DEMARCHE ENVIRONNEMENTALE : ATELIER & BUREAUX
% Texte exact de la base de données
% ============================================

\subsection{DÉMARCHE ENVIRONNEMENTALE : ATELIER \& BUREAUX}

\begin{tcolorbox}[
    enhanced,
    colback=ecoFond,
    colframe=ecoVert,
    fonttitle=\bfseries\large,
    coltitle=white,
    title={\faRecycle\hspace{0.5em}Démarche environnementale : Atelier \& Bureaux},
    colbacktitle=ecoVertFonce,
    rounded corners,
    boxrule=2pt,
    left=8pt, right=8pt, top=8pt, bottom=8pt
]

Nous attachons une importance toute particulière au traitement de nos déchets et globalement à l'empreinte énergétique de l'entreprise sur son milieu. Notre activité est faiblement productrice de déchets, et ceux-ci sont généralement faciles à recycler par les filières classiques de traitement.

Cette double démarche (déchets/empreinte énergétique) s'applique dans tous les secteurs de notre activité comme l'atteste notre schéma d'organisation et de gestion des déchets (SOGED) à l'atelier ou sur nos chantiers.

\end{tcolorbox}

\vspace{0.5cm}

% ---- ACTIONS CONCRETES ----
\begin{tcolorbox}[
    enhanced,
    colback=white,
    colframe=ecoVert,
    fonttitle=\bfseries,
    coltitle=white,
    title={\faCheckCircle\hspace{0.3em}Actions concrètes},
    colbacktitle=ecoVert,
    rounded corners,
    boxrule=1.5pt,
    left=8pt, right=8pt, top=8pt, bottom=8pt
]

\textbf{ Dans notre ateliers et bureau d'étude : }

\vspace{0.3cm}

\begin{itemize}[leftmargin=*, itemsep=4pt]

    \item Réduction de notre consommation de papier, consommables.

    \item Privilégie la digitalisation des documents aux impressions, si impression nécessaire les impressions noir et blanc seront privilégiées.

    \item Une seule imprimante centralisée. Nos contrats de location de matériel bureautique prévoient la reprise des consommables vides et des anciens matériels.

    \item Investissement informatique permettant de privilégier le stockage et la transmission des documents par voie numérique.

    \item L'éclairage est optimisé et la lumière naturelle est privilégiée.

    \item Trie de nos déchets (papier, carton, plastique, bureautique,…).

    \item Chauffage de notre atelier par chaudière à bois alimentée par nos résidus de fabrication et de chantier.

    \item Le choix de fournisseurs locaux est privilégié.

\end{itemize}

\end{tcolorbox}

\vspace{0.5cm}

% ---- IMAGE ACTION CONCRETE ----
\begin{tcolorbox}[
    enhanced,
    colback=ecoFond,
    colframe=ecoVert,
    fonttitle=\bfseries,
    coltitle=white,
    title={\faImage\hspace{0.3em}Action concrète : Disposition de nos systèmes de tri dans notre atelier},
    colbacktitle=ecoVert,
    rounded corners,
    boxrule=1.5pt,
    left=8pt, right=8pt, top=8pt, bottom=8pt
]

\begin{center}
\includegraphics[width=0.8\textwidth]{../images/HQE_chantier}
\end{center}

\end{tcolorbox}

\vspace{0.5cm}

% ---- TRI SELECTIF ----
\begin{tcolorbox}[
    enhanced,
    colback=white,
    colframe=ecoVert,
    fonttitle=\bfseries,
    coltitle=white,
    title={\faTrash\hspace{0.3em}Notre démarche de tri-sélectif},
    colbacktitle=ecoVert,
    rounded corners,
    boxrule=1.5pt,
    left=8pt, right=8pt, top=8pt, bottom=8pt
]

Dans sa démarche écologique, Bois \& Techniques tri aujourd'hui ses déchets à la source grâce à des poubelles de tri et des bennes sélectives. Permettant leur traitement selon leur nature et leur destination.

\vspace{0.4cm}

\begin{itemize}[leftmargin=*, itemsep=4pt]

    \item Benne pour : bois sains, lattes, chutes de poutres, volige, cartons non souillés : utilisé comme moyen de chauffage de l'atelier interne.

    \item Benne pour déchets bois traités et bois peint évacuation vers um centre de traitement \underline{SCHROLL}.

    \item Lieu de stockage pour déchets industriels spéciaux : Les mastics, silicones, colles, pots de peinture et Epoxy, mousse Polyuréthane expansée, contenants souillés, graisse, chiffons, et autres déchets. stocké dans un local avec cuve de rétention. Puis évacuation et traitement chez TREDI ou déchetterie de la commune de SOULTZ.

    \item Silo pour copeaux et sciure (Fabrication de briquette).

    \item Benne pour déchets inerte : tuiles, maçonnerie, divers gravats… évacuation par \underline{SCHROLL} puis transformation pour réutilisation comme remblaiement.

    \item Entreposage des métaux triés : acier, aluminium, zinc, cuivre, visserie, équerre, pointe,… puis recyclage par broyage à destination des fonderies.

    \item Benne mélange : emballage, polyane, isolant, récupérés par la société \underline{SCHROLL} pour stockage et revalorisation après extraction.

    \item Poubelle déchets ménagers collectée par la municipalité.

\end{itemize}

\end{tcolorbox}

\vspace{0.5cm}

% ---- DIMINUER DECHETS ----
\begin{tcolorbox}[
    enhanced,
    colback=white,
    colframe=ecoVert,
    fonttitle=\bfseries,
    coltitle=white,
    title={\faArrowDown\hspace{0.3em}Diminuer sa production de déchets en atelier},
    colbacktitle=ecoVert,
    rounded corners,
    boxrule=1.5pt,
    left=8pt, right=8pt, top=8pt, bottom=8pt
]

\begin{itemize}[leftmargin=*, itemsep=4pt]

    \item Prédimensionnement de pièces pour limiter les déchets de matière.

    \item Livraisons et stockage mieux ordonnés, pour prévenir la casse, les chutes et les détériorations.

    \item Choix de produit moins emballés.

\end{itemize}

\end{tcolorbox}

\vspace{0.5cm}

% ---- SENSIBILISATION ----
\begin{tcolorbox}[
    enhanced,
    colback=ecoVertClair,
    colframe=ecoVertFonce,
    fonttitle=\bfseries,
    coltitle=white,
    title={\faUsers\hspace{0.3em}Action de sensibilisation},
    colbacktitle=ecoVertFonce,
    rounded corners,
    boxrule=1.5pt,
    left=8pt, right=8pt, top=8pt, bottom=8pt
]

A pour objectif de favoriser l'éducation du personnel de l'entreprise au tri des déchets et aux systèmes de tri mis en place.

Chaque nouvelle installation de tri ou méthode est expliquée collectivement et des rappels individuels sont effectués à chaque constatation du non-respect de ces engagements.

\end{tcolorbox}
    

    
    
    
    % Inclusion du fichier spécial pour la démarche environnementale chantiers
    % ============================================
% DEMARCHE ENVIRONNEMENTALE : SUR LES CHANTIERS
% Texte exact de la base de données
% ============================================

\newpage
\subsection{DÉMARCHE ENVIRONNEMENTALE : SUR LES CHANTIERS}

\begin{tcolorbox}[
    enhanced,
    colback=white,
    colframe=ecoVert,
    fonttitle=\bfseries\large,
    coltitle=white,
    title={\faHardHat\hspace{0.5em}Démarche environnementale : sur les chantiers},
    colbacktitle=ecoVertFonce,
    rounded corners,
    boxrule=2pt,
    left=8pt, right=8pt, top=8pt, bottom=8pt
]

L'entreprise Bois \& Techniques a mis en place depuis de nombreuses années, pour réduire ses coûts d'élimination, un tri sélectif de ses déchets en sortie de production, sur chantier et en retour de chantier.

Nous avons 3 principes de tri sur les chantiers en fonction du type, de la taille et de la quantité de déchet à traiter.

Notre activité en construction neuve ne génère presque pas de déchet sur chantier, puisque l'usinage se réalise en atelier.

\end{tcolorbox}

\vspace{0.5cm}

% ---- CAS N1 ----
\begin{tcolorbox}[
    enhanced,
    colback=white,
    colframe=ecoVert,
    fonttitle=\bfseries,
    coltitle=white,
    title={\faUsers\hspace{0.3em}Cas n°1 : Notre démarche dans un cadre de tri collectif},
    colbacktitle=ecoVert,
    rounded corners,
    boxrule=1.5pt,
    left=8pt, right=8pt, top=8pt, bottom=8pt
]

\begin{itemize}[leftmargin=*, itemsep=4pt]

    \item Les différents moyen de gestion et de tri des déchets présent sur le chantier (bennes, poubelles, Big bag, zone de stockage, …) seront identifiés et utilisés par nos équipes dans le respect de leur fonction (DI, DIB, DID).

    \item Nos équipes sont sensibilisées pour répartir correctement leurs déchets dans les différentes bennes par catégorie (inertes, industriels banals, industriels dangereux) et par catégorie de matériaux (métaux, plastiques, bois et dérivés, liquides, …)

    \item Dans le cas où certains types de déchets créés par nos équipes sur le chantier ne seraient pas compatibles avec les systèmes collectifs de récupération. Nous nous chargerons de leur évacuation vers notre atelier et mettrons en œuvre notre processus de tri interne.

\end{itemize}

\end{tcolorbox}

\vspace{0.5cm}

% ---- CAS N2 ----
\begin{tcolorbox}[
    enhanced,
    colback=white,
    colframe=ecoVert,
    fonttitle=\bfseries,
    coltitle=white,
    title={\faTruck\hspace{0.3em}Cas n°2 : Notre démarche de tri autonome pour gros volume},
    colbacktitle=ecoVert,
    rounded corners,
    boxrule=1.5pt,
    left=8pt, right=8pt, top=8pt, bottom=8pt
]

\textbf{ SI : ABSENCE DE GESTION DES DÉCHETS COLLECTIVE / QUANTITÉ DE DÉCHET PRODUIT IMPORTANTE / VARIÉTÉ DE TYPE DE DÉCHET. }

\vspace{0.3cm}

\begin{itemize}[leftmargin=*, itemsep=4pt]

    \item Une aire d'entreposage des déchets sera définie et le nombre de bennes découlera de la spécificité des travaux et de la quantité de déchets générées.

    \item L'aire d'entreposage sera placée à bonne distance des zones sensibles pour prévenir tout risque de contamination (sol, personnel, …).

    \item Elle sera balisée, rangée régulièrement et maintenu dans un état de propreté afin d'en faciliter l'accès et la récupération des bennes par nos prestataires.

    \item Les différentes bennes tenues à disposition, par un prestataire agréé, pour le tri des déchets seront identifiées par la mise en place de pictogrammes ou de codes couleurs représentants les types de déchets qu'elles peuvent contenir.

    \item Collecte des déchets par notre prestataire qui les valorisera dans les conditions légales, c'est-à-dire par réemploi, recyclage ou transformation en énergie, à l'exclusion de tout autre mode d'élimination, ou par nos équipes dans le cas de déchets spécifiques et respect de la traçabilité des produits dangereux.

    \item Nos équipes sont évidemment sensibilisées pour répartir correctement leurs déchets dans les différentes bennes par catégorie (inertes, industriels banals, industriels dangereux) et par catégorie de matériaux (métaux, plastiques, bois et dérivés, liquides, …).

    \item Mise en place de moyen de tri initial à côté de nos zones de travail (sacs poubelle, Big bag, …) afin de mieux gérer le flux de déchets vers les bennes.

    \item Entretien et nettoyage régulier des zones de travail.

\end{itemize}

\end{tcolorbox}

\vspace{0.5cm}

% ---- INTERDICTIONS ----
\begin{tcolorbox}[
    enhanced,
    colback=red!5,
    colframe=ecoRouge,
    fonttitle=\bfseries,
    coltitle=white,
    title={\faBan\hspace{0.3em}Proscrit},
    colbacktitle=ecoRouge,
    rounded corners,
    boxrule=1.5pt,
    left=8pt, right=8pt, top=8pt, bottom=8pt
]

\begin{itemize}[leftmargin=*, itemsep=4pt]

    \item Brûler les déchets.

    \item Enfouir des déchets.

    \item Mettre en dépôt sauvage.

\end{itemize}

\end{tcolorbox}

\vspace{0.5cm}

% ---- IMAGE ORGANISATION ----
\begin{tcolorbox}[
    enhanced,
    colback=white,
    colframe=ecoVert,
    fonttitle=\bfseries,
    coltitle=white,
    title={\faImage\hspace{0.3em}Organisation de la gestion des déchets sur un chantier},
    colbacktitle=ecoVert,
    rounded corners,
    boxrule=1.5pt,
    left=8pt, right=8pt, top=8pt, bottom=8pt
]

\begin{center}
\includegraphics[width=0.9\textwidth]{../images/orga_chantier_gestion}
\end{center}

\end{tcolorbox}

\vspace{0.5cm}

% ---- CAS N3 ----
\begin{tcolorbox}[
    enhanced,
    colback=white,
    colframe=ecoVert,
    fonttitle=\bfseries,
    coltitle=white,
    title={\faBox\hspace{0.3em}Cas n°3 : Notre démarche de tri autonome pour petit volume},
    colbacktitle=ecoVert,
    rounded corners,
    boxrule=1.5pt,
    left=8pt, right=8pt, top=8pt, bottom=8pt
]

\textbf{ SI : ABSENCE DE GESTION DES DÉCHETS COLLECTIVE / QUANTITÉ DE DÉCHET PRODUIT FAIBLE. }

\vspace{0.3cm}

Nos équipes utiliseront le système de tri et de récupération des déchets mis en place par notre société :

\vspace{0.3cm}

\begin{itemize}[leftmargin=*, itemsep=4pt]

    \item Tri des bois en deux catégorie (bois sain, et bois traités ou peint) dans nos remorques ou dans les parties de nos camionnettes prévues pour le transport de bois.

    \item Récupération des métaux (visserie, pointe, ferrure, …) dans les seaux et caisses prévus à cet effet.

    \item Tri des matériaux facilement recyclable dans sac polyane (bouteille plastique, carton, verre).

    \item Récupération des résidus de chantier non inertes et non dangereux dans différents Big bag et sac polyane (en fonction des quantités) (gravât, plâtre, isolant, polystyrène, … etc.).

    \item Récupération des résidus de chantier dangereux dans différents Big-bags et sac polyane (adaptés à cette effet) (bois souillé, peinture, Epoxy, huile, …).

    \item Ces déchets seront évacués en fin de journée ou périodiquement par nos soins (dans nos véhicules ou à l'aide de remorques) pour être ensuite répartis dans les aires de stockage de notre atelier avant d'être distribué vers les centres de retraitement de nos prestataires ou de la commune de SOULTZ.

    \item Nos équipes sont évidemment sensibilisées pour répartir correctement leurs déchets dans les différentes moyen de tri mis à leur disposition (remorque, sac, Big-bag, caisse, seau,…) par catégorie (inertes, industriels banals, industriels dangereux) et par catégorie de matériaux (métaux, plastiques, bois et dérivés, liquides, …)

\end{itemize}

\end{tcolorbox}

\vspace{0.5cm}

% ---- Fiche interne ----
\begin{tcolorbox}[
    enhanced,
    colback=ecoVertClair,
    colframe=ecoVertFonce,
    fonttitle=\bfseries,
    coltitle=white,
    title={\faClipboard\hspace{0.3em}Fiche interne},
    colbacktitle=ecoVertFonce,
    rounded corners,
    boxrule=1.5pt,
    left=8pt, right=8pt, top=8pt, bottom=8pt
]

\textbf{ Pour chaque projet nous utilisons une fiche interne, pour préparer la gestion de nos déchets. }

\end{tcolorbox}
    

    





\newpage
\section{ CHANTIERS DE RÉFÉRENCE }

    





\newpage
\section{ LISTE DES MATERIAUX MIS EN OEUVRE }

    
    
    
    % Inclusion du fichier spécial pour matière première de qualité certifiée
    \subsection{ UNE MATIERE PREMIERE DE QUALITE CERTIFIEE }
    % ============================================
% UNE MATIERE PREMIERE DE QUALITE CERTIFIEE
% Texte exact de la base de données avec logos
% ============================================

\begin{tcolorbox}[
    enhanced,
    colback=white,
    colframe=ecoVert,
    fonttitle=\bfseries\Large,
    coltitle=ecoVertFonce,
    title={},
    rounded corners,
    boxrule=0pt,
    left=5pt, right=5pt, top=5pt, bottom=5pt
]

% Ligne 1 : Label Vert + texte
\begin{minipage}[c]{0.15\textwidth}
\centering
\includegraphics[width=0.9\linewidth]{../images/logo_excell_label_vert.jpeg}\end{minipage}
\hfill
\begin{minipage}[c]{0.82\textwidth}
Utilisation de bois certifiés, de colle à base aqueuse ainsi que les traitements de classe 2 (label vert -- SARPECO 850).
\end{minipage}

\vspace{0.4cm}
\hrule
\vspace{0.4cm}

% Ligne 2 : PEFC + FSC + texte
\begin{minipage}[c]{0.07\textwidth}
\centering
\includegraphics[width=\linewidth]{../images/logo_pefc2 (1).png}
\end{minipage}
\hfill
\begin{minipage}[c]{0.07\textwidth}
\centering
\includegraphics[width=\linewidth]{../images/logo_fsc.png}
\end{minipage}
\hfill
\begin{minipage}[c]{0.82\textwidth}
Les approvisionnements sont faits dans le respect de l'environnement avec une politique d'achats rigoureuse des bois à provenance certifié et une éco certification des bois locaux.
\end{minipage}

\vspace{0.4cm}
\hrule
\vspace{0.4cm}

% Ligne 3 : Achetons Local + texte
\begin{minipage}[c]{0.15\textwidth}
\centering
\includegraphics[width=0.9\linewidth]{../images/logo_achetons_local.jpg}
\end{minipage}
\hfill
\begin{minipage}[c]{0.82\textwidth}
Fournisseurs locaux et français de préférence.
\end{minipage}

\vspace{0.4cm}
\hrule
\vspace{0.4cm}

% Ligne 4 : Excell Zone Verte + texte
\begin{minipage}[c]{0.15\textwidth}
\centering
\includegraphics[width=0.9\linewidth]{../images/logo_excell_zone_verte.png}
\end{minipage}
\hfill
\begin{minipage}[c]{0.82\textwidth}
Matériaux testés en laboratoire et l'ensemble de nos panneaux OSB sont de classe E1 ($\leq$ 8mg/100g) ou garantis sans formaldéhyde.
\end{minipage}

\end{tcolorbox}

\vspace{0.5cm}

% Encadré santé
\begin{tcolorbox}[
    enhanced,
    colback=ecoFond,
    colframe=red!70!black,
    fonttitle=\bfseries,
    coltitle=white,
    title={\faHeart\hspace{0.3em}Santé et environnement},
    colbacktitle=red!70!black,
    rounded corners,
    boxrule=1.5pt,
    left=10pt, right=10pt, top=10pt, bottom=10pt
]

Nous privilégions des produits moins nocifs pour la santé limitant les émissions potentielles (émissions chimiques et radioactives, comportement vis à vis des moisissures et des bactéries, émissions d'odeurs).

\end{tcolorbox}
    

    
    
    
    % Tableau stylé pour Fixation et assemblage
    \begin{tcolorbox}[
        enhanced,
        colback=white,
        colframe=ecoMarron,
        fonttitle=\bfseries\large,
        coltitle=white,
        title={\faIcon{cogs}\hspace{0.3em}Fixation et Assemblage},
        colbacktitle=ecoMarron,
        rounded corners,
        boxrule=2pt,
        left=3pt, right=3pt, top=5pt, bottom=5pt
    ]
    
    \renewcommand{\arraystretch}{1.3}
    \small
    \rowcolors{2}{ecoFond}{white}
    \begin{tabular}{p{4.5cm}p{4.5cm}p{3cm}p{2cm}}

\rowcolor{ecoMarron!20}\textbf{Nature des éléments} & \textbf{Marque, type, performance} & \textbf{Provenance} & \textbf{Doc en annexe} \\

Boulons galvanisés à chaud Ø16 et Ø20 & Classe 6.8. Filetage partiel & France & OUI \\

Visserie électrozinguée & BERNER & France & OUI \\

Vis Ø8 & EUROTEC & Allemagne & OUI \\

Equerres de fixation et sabots galvanisés BEA & BEA & France & OUI \\

Pointes d’ancrage crantées & BEA 4/50 & France & OUI \\

Ferrures mécano-soudées & MTR & France - Alsace & OUI \\

Vis anti-fendage & SFS – EUROTEC & France - Allemagne & OUI \\

Pointes brutes & GUNEBO & Suède : Suède & OUI \\

Pointes galvanisées à chaud & GUNEBO & France & OUI \\

Chevilles maçonnerie & HILTI HRD UGT Ø 10 – Ø 14 & France & OUI \\

Goujons d’ancrage & HILTI HSAK Ø 12 – Ø 16 & France & OUI \\

Vis inox & BERNER A2 torx & Allemagne & OUI \\

Vis inox & EUROTEC A4 torx & France & OUI \\

Boulons TRCC 8 – Ø 10 – Ø 12 & SCHMERBER &  & OUI \\

\end{tabular}
    
    \end{tcolorbox}
    

    
    
    
    % Inclusion du fichier spécial pour méthodologie de traitement
    % ============================================
% Méthodologie de traitement
% Bloc único contenant toute la méthodologie
% ============================================

% Bloc principal unique pour toute la méthodologie
\begin{tcolorbox}[
    enhanced,
    colback=white,
    colframe=ecoVert,
    fonttitle=\bfseries\large,
    coltitle=white,
    title={\faIcon{spray-can}\hspace{0.3em}Méthodologie de traitement},
    colbacktitle=ecoVert,
    rounded corners,
    boxrule=2pt,
    left=10pt, right=10pt, top=10pt, bottom=10pt
]

% Texte introductif
\textbf{ Traitement curatif et préventif par injections et double pulvérisation de l'ensemble des bois de charpente et solivage intérieurs selon les prescriptions techniques en vigueur : }

\vspace{0.3cm}

% Préparation des surfaces
\noindent
\textbf{\textcolor{traitBleu}Préparation des surfaces}
\begin{itemize}[leftmargin=*, itemsep=2pt]

    \item Sondage

    \item Décapage des bois atteints (Bûchage)

    \item Brossage

    \item Dépoussiérage

\end{itemize}

\vspace{0.3cm}

% Traitement en profondeur - deux colonnes
\noindent
\textbf{\textcolor{traitBleu}Traitement en profondeur} \\
\vspace{0.1cm}

\noindent
\begin{minipage}[t]{0.48\textwidth}
\vspace{0pt}

\textbf{\small Grosses pièces}
\begin{itemize}[leftmargin=*, itemsep=2pt]
    \footnotesize

    \item Percement des puits d'injection tous les \textbf{20cm (5 à 6 par ML)}

    \item Mise en place d'injecteurs anti retour sans tête restant définitivement dans les éléments traités

    \item Pièces encastrées dans la maçonnerie : injections doublées et orientées vers la maçonnerie

    \item Injections profondes en basse pression \textbf{20 à 40 g}

    \item Application en surface de XILIX 3000P sur l'ensemble des bois en deux couches \textbf{(100 à 300g par m²)}

\end{itemize}

\end{minipage}
\hfill
\begin{minipage}[t]{0.48\textwidth}
\vspace{0pt}

\textbf{\small Chevrons}
\begin{itemize}[leftmargin=*, itemsep=2pt]
    \footnotesize

    \item Percement des puits d'injection tous les \textbf{30cm (3 à 4 par ML)}

    \item Mise en place d'injecteurs anti retour sans tête restant définitivement dans les éléments traités

    \item Injections profondes en basse pression \textbf{20 à 40 g}

    \item Application en surface de XILIX 3000P sur l'ensemble des bois en deux couches \textbf{(100 à 300g par m²)}

\end{itemize}

\end{minipage}

\end{tcolorbox}
    

    
    
    
    % Tableau stylisé pour Traitement Préventif
    \begin{tcolorbox}[
        enhanced,
        colback=white,
        colframe=ecoVert,
        fonttitle=\bfseries\large,
        coltitle=white,
        title={\faIcon{shield-alt}\hspace{0.3em}Traitement préventif des bois},
        colbacktitle=ecoVert,
        rounded corners,
        boxrule=2pt,
        left=3pt, right=3pt, top=5pt, bottom=5pt
    ]
    
    \renewcommand{\arraystretch}{1.3}
    \small
    \rowcolors{2}{ecoFond}{white}
    \begin{tabular}{p{4cm}p{6cm}p{2.5cm}p{1.5cm}}

\rowcolor{ecoVert!20}\textbf{Nature des éléments} & \textbf{Marque, type, performance} & \textbf{Provenance} & \textbf{Doc en annexe} \\

Traitement SARPECO 850 & Label Vert - SARPECO 850 & France & OUI \\

\end{tabular}
    
    \end{tcolorbox}
    

    
    
    
    % Tableau stylisé pour Traitement Curatif
    \begin{tcolorbox}[
        enhanced,
        colback=white,
        colframe=ecoVert,
        fonttitle=\bfseries\large,
        coltitle=white,
        title={\faIcon{hammer}\hspace{0.3em}Traitement curatif des bois},
        colbacktitle=ecoVert,
        rounded corners,
        boxrule=2pt,
        left=3pt, right=3pt, top=5pt, bottom=5pt
    ]
    
    \renewcommand{\arraystretch}{1.3}
    \small
    \rowcolors{2}{ecoFond}{white}
    \begin{tabular}{p{4cm}p{6cm}p{2.5cm}p{1.5cm}}

\rowcolor{ecoVert!20}\textbf{Nature des éléments} & \textbf{Marque, type, performance} & \textbf{Provenance} & \textbf{Doc en annexe} \\

Traitement XILIX 3000P & Label Vert - CTB + - PHASE AQUEUSE ; CLASSE 1/2/3A CERTIFIE CTB-P+ / SOCOTEC & France & OUI \\

\end{tabular}
    
    \end{tcolorbox}
    

    




\newpage
% ANNEXES
\section{ANNEXES}

\end{document}