% ============================================
% SECURITE ET SANTE SUR LES CHANTIERS
% ============================================

\begin{tcolorbox}[
    enhanced, colback=white, colframe=ecoBleu, fonttitle=\bfseries\small, coltitle=white,
    title={\faIcon{cog}\hspace{0.2em}Organisation de production},    colbacktitle=ecoBleu, rounded corners, boxrule=1pt, left=4pt, right=4pt, top=3pt, bottom=3pt
]
\footnotesize

Notre atelier est organisé de façon lean management en privilegiant l'efficience afin de faciliter la circulation des pièces de bois au cours des différentes phases de travail. Les zones de stockage, de traçage, d'usinage et de stockage avant expédition sont prévues.

\end{tcolorbox}

\vspace{0.4cm}

\begin{tcolorbox}[
    enhanced, colback=white, colframe=ecoBleu, fonttitle=\bfseries\small, coltitle=white,
    title={\faIcon{user-plus}\hspace{0.2em}Accueil des nouveaux salariés},
    colbacktitle=ecoBleu, rounded corners, boxrule=1pt, left=4pt, right=4pt, top=3pt, bottom=3pt
]
\footnotesize
Chaque nouvel arrivant (stagiaire, intérimaire, salarié) se voit fournir, dès son arrivée, un livret d'accueil rappelant les bases de sécurité et les bons comportements à avoir (OPP BTP), une « fiche d'accueil » est rédigée avec lui, afin de faciliter son suivi médical en cas d'accident et de le sensibiliser aux dangers du chantier et les types d'EPI mis à sa disposition.
\end{tcolorbox}

\vspace{0.4cm}

\begin{tcolorbox}[
    enhanced, colback=white, colframe=ecoBleu, fonttitle=\bfseries\small, coltitle=white,
    title={\faIcon{certificate}\hspace{0.2em}Habilitations et compétences réglementaires},    colbacktitle=ecoBleu, rounded corners, boxrule=1pt, left=4pt, right=4pt, top=3pt, bottom=3pt
]
\footnotesize

L'ensemble du personnel intervenant sur les chantiers dispose des habilitations et certifications réglementaires nécessaires à l'exécution des travaux en toute sécurité. Les équipes sont régulièrement formées et maintenues à niveau afin de garantir la conformité aux exigences en vigueur et la maîtrise des risques liés aux interventions.

À ce titre, les opérateurs sont notamment titulaires des habilitations suivantes :
\begin{itemize}[leftmargin=1.5em, itemsep=2pt]
    \item CACES Levage
    \item CACES Conduite de pont roulant
    \item CACES Nacelle
    \item SST (Sauveteur Secouriste du Travail)
\end{itemize}

\end{tcolorbox}

\vspace{0.4cm}

\begin{tcolorbox}[
    enhanced, colback=white, colframe=ecoBleu, fonttitle=\bfseries\small, coltitle=white,
    title={\faIcon{coffee}\hspace{0.2em}Confort de travail},    colbacktitle=ecoBleu, rounded corners, boxrule=1pt, left=4pt, right=4pt, top=3pt, bottom=3pt
]
\footnotesize

Que ce soit en atelier ou sur chantier, nous mettons à la disposition de notre personnel \textbf{une salle de repos} avec : point d'eau, machine à café, réfrigérateur, micro-onde, table et chaises, vestiaire, douche, sanitaires\ldots

\end{tcolorbox}

\vspace{0.4cm}

\begin{tcolorbox}[
    enhanced, colback=ecoFond, colframe=ecoRouge, fonttitle=\bfseries, coltitle=white,
    title={\faIcon{shield-alt}\hspace{0.3em}Sécurité et santé sur les chantiers},
    colbacktitle=ecoRouge, rounded corners, boxrule=1.5pt, left=5pt, right=5pt, top=4pt, bottom=4pt
]
\small
Afin d'offrir des garanties de sécurité au public et nos salariés, notre atelier, \textbf{construit en 2010}, respecte l'ensemble des normes de construction et de prévention/lutte/accès incendie et explosion. Nos équipes ont à leur disposition les moyens de travailler en sécurité tout en préservant leur santé. Ils participent régulièrement à des formations sur les bonnes pratiques sécuritaires et le bon usage de nos machines (tailles et levages).
\end{tcolorbox}

\vspace{0.4cm}

\begin{tcolorbox}[
    enhanced, colback=white, colframe=ecoRouge, fonttitle=\bfseries\small, coltitle=white,
    title={\faIcon{tasks}\hspace{0.2em}Concrètement},    colbacktitle=ecoRouge, rounded corners, boxrule=1pt, left=4pt, right=4pt, top=3pt, bottom=3pt
]
\footnotesize

\begin{itemize}[leftmargin=*, itemsep=0pt, parsep=0pt, topsep=0pt]
    \item NORMES et CERTIFICATIONS MACHINES (fixes et portatives).
    \item NORMES ASPIRATION.
    \item SECURITE INCENDIE. (Prévention, lutte, accès).
    \item NORMES ELECTRIQUE.
    \item ACCESSIBILITE AUX PERSONNES EN SITUATION DE HANDICAP.
    \item HYGIENE, SANTE et CONFORT.
    
\end{itemize}

\end{tcolorbox}

\vspace{0.4cm}

\begin{tcolorbox}[
    enhanced, colback=white, colframe=ecoRouge, fonttitle=\bfseries\small, coltitle=white,
    title={\faIcon{check-double}\hspace{0.2em}Vérification périodique},    colbacktitle=ecoRouge, rounded corners, boxrule=1pt, left=4pt, right=4pt, top=3pt, bottom=3pt
]
\footnotesize

Échelles, harnais, stop-chute, équipement de levage, ligne de vie, installation électrique, détection incendie, RIA, compresseur, chariot élévateur, extincteur, machine atelier\ldots

\end{tcolorbox}

\vspace{0.4cm}

\begin{tcolorbox}[
    enhanced, colback=white, colframe=ecoRouge, fonttitle=\bfseries\small, coltitle=white,
    title={\faIcon{exclamation-triangle}\hspace{0.2em}La sécurité s'affiche partout},
    colbacktitle=ecoRouge, rounded corners, boxrule=1pt, left=4pt, right=4pt, top=3pt, bottom=3pt
]
\footnotesize
De nombreuses affiches issues des organismes de prévention reconnus, tels que l'INRS et l'OPPBTP, sont disposées aux endroits stratégiques de nos bureaux et de notre atelier.

Puisque la pédagogie est l'art de la répétition nous avons privilégié des modèles avec des graphismes et slogan plus ancien pour rappeler à nos charpentiers celle qui se trouvaient déjà autour d'eux dans leur lieu de formation (collège, lycée, CFA,\ldots).
\end{tcolorbox}
